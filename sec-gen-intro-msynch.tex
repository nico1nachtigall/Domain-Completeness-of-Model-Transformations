Basically, model synchronisations $\MSynch\colon \Delta_{D_1} \to \Delta_{D_2}$ are performed based on model transformations where model updates $\delta \in \Delta_{D_1}$ in domain $D_1$ are propagated (mapped) to (model updates $\delta' \in \Delta_{D_2}$ in) domain $D_2$ by performing model transformations.
Therefore, the classification of model transformations from \cref{sec-gen-intro-mt} can also be applied to model synchronisations, i.e., model synchronisations may be
\begin{enumerate*}
\item endogenous or exogenous,
\item horizontal or vertical,
\item based on textual or visual transformation specifications,
\item based on declarative or operational transformation specifications, and
\item propagations of updates from model-to-model, model-to-text, text-to-model or text-to-text.
\end{enumerate*}

According to \cref{fig:sec-gen-intro-msynch:mt_msynch} (b), a model update in some domain $D$ relates model $M_1 \in \Lang(D)$ from before the update with model $M_2 \in \Lang(D)$ from after performing the update and therefore, the update documents the changes to $M_1$ which have led to $M_2$.
Thus, a model update relates models that are expressed in the same DSL $\Lang(D)$, i.e., models $M_1$ and $M_2$ conform to the same meta-model in domain $D$.
\cref{fig:sec-gen-intro-msynch:mt_msynch} (c) illustrates the execution of a model synchronisation that takes a model update $\delta$ in source domain $D_1$ from model $M_1$ to model $M_2$ as input together with a correspondence (\code{Corr}) which interrelates model $M_1$ with model $M'_1$ in target domain $D_2$ to which the update should be propagated.
The execution outputs an update $\delta'$ in target domain $D_2$ from model $M'_1$ to model $M'_2$ together with a correspondence (\code{Corr'}) which relates model $M_2$ with model $M'_2$.
The model synchronisation (MSynch) is performed based on executing the model transformation specification.

According to \cref{sec-gen-intro-mt}, we focus on model-to-model synchronisations based on TGGs that allow visual, declarative transformation specifications.
We review basic concepts and notions in \cref{sec-tgg,sec-msynch-tgg}.
Beside the model-to-model propagation of model updates from UML class diagrams to relational database models in \cref{sec-msynch-tgg}, we present software synchronisations, i.e., text-to-model(text) propagations of updates from source code to visual models (or source code again) in \cref{sec-compl-software-synch}.
