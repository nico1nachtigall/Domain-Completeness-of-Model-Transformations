As illustrated in \cref{fig:sec-compl-software-trans:softwaretrans}, our approach of translating programs of a programming language $\Lang^S$ is to translate their derivation trees as induced by the word grammar $G^S$ of $\Lang^S$.
Each derivation tree is represented by a typed attributed graph in the source graph language $\mathcal{VL}^S$ of derivation trees and translated by performing a model transformation based on a given triple graph grammar.
The result is a typed attributed graph in the target modelling (graph) language $\mathcal{VL}^T$.
Furthermore, we assume that the source (target) graph language is defined by a type graph $\TG^S$ ($\TG^T$) together with a set $C^S$ ($C^T$) of graph contraints.

\begin{definition}[Graph Language over Type Graph \& Constraints]
\index{language!$\Lang(C)$ over graph constraints}
Given a type graph $\TG$ and a set $C$ of graph constraints that are typed over $\TG$.
Then, the \emph{graph language over $\TG$ and $C$} is given by $\Lang(\TG,C)=\{G \mid \text{graph }G\text{ is typed over }\TG,G \models C\}$ all graphs $G$ that are typed over $\TG$ and that satisfy the constraints $C$.
\envEndMarker
\end{definition}

The translation is syntactically complete, if all derivation trees of $G^S$ can be completely translated.
Intuitively, a derivation tree is completely translated, if all nodes and edges of the tree are translated exactly once.
Since, the translation of derivation trees is based on the translation of their graph representations, it is required that the language $\Der(G^S)$ of derivation trees of  $G^S$ is isomorphic to the graph language $\Lang(\TG^S,C^S)$ of their graph representations (cf. \cref{sec-compl-software-trans-mapping,thm:sec-compl-software-trans:equ_lang_der}).
Thus, each tree in $\Der(G^S)$ is represented by exactly one graph in $\Lang(\TG^S,C^S)$ and each graph in $\Lang(\TG^S,C^S)$ represents exactly one derivation tree in $\Der(G^S)$.

\begin{definition}[Isomorphism of Derivation Tree Languages]
\label{def:sec-compl-software-trans-prob:iso}
\index{language!of derivation trees!isomorphism}
Given a context-free word grammar $G$ with induced language $\Der(G)$ of derivation trees.
Let $\Lang(\TG,C)$ be a graph language of derivation trees over type graph $\TG$ and graph constraints $C$.
\emph{Languages $\Der(G)$ and $\Lang(\TG,C)$ are isomorphic}, if there exists a bijection $m\colon \Der(G) \to \Lang(\TG,C)$ between both.
\envEndMarker
\end{definition}

By \cref{def:comp_soft_trans}, a software translation is syntactically complete if each source derivation graph $G^S \in \Lang(\TG^S,C^S)$ can be completely translated via a model transformation sequence in the sense that all elements of the graph are translated exactly once by changing their translation attributes from $\False$ to $\True$.

\begin{definition}[Syntactical Completeness of Software Translations]
\label{def:comp_soft_trans}
\index{software transformation!completeness}
Given a context-free word grammar $G^S$ for source language $\Lang^S$ with induced language $\Der(G^S)$ of derivation trees.
Let $\TG=(\TG^S \gets \TG^C \to \TG^T)$ be a triple type graph with $\Lang(\TG^S,C^S)$ being the graph language of derivation trees isomorphic to $\Der(G^S)$. 
Furthermore, let $\TGG=(\varnothing,\TR)$ be a triple graph grammar typed over $\TG$ with derived forward translation rules $\TR_\FT$ that specifies the model transformation of graphs in $\Lang(\TG^S,C^S)$ into graphs of target graph language $\Lang(\TG^T,C^T)$ based on forward translation rules $\TR_\FT$.
\emph{The translation is syntactically complete}, if for each derivation graph $G^S \in \Lang(\TG^S,C^S)$ there is a model transformation sequence $(G^S,G_0 \Trans{\tr^*_\FT} G_n,G^T)$ based on forward translation rules $\TR_\FT$ with $G_0=(\Att^\False(G^S) \gets \varnothing \to \varnothing)$ and $G_n=(\Att^\True(G^S) \gets G^C \to G^T)$.\envEndMarker
\end{definition}

From the setting for translations in \cref{fig:sec-compl-software-trans:softwaretrans} and from \cref{def:comp_soft_trans,fact:comp_soft_trans}, the following problems arise concerning the completeness and correctness of software translations.

\begin{problem}[Syntactical Completeness \& Correctness of Software Translations]
\label{prob:scst}
\begin{enumerate}
  \item \label{item:scst1} Syntactical Completeness \& Correctness: Given a context-free word grammar $G$ with induced language $\Der(G)$ of derivation trees.
  How to derive a graph language $\Lang(\TG,C)$ of derivation trees from $G$ that is isomorphic to $\Der(G)$?
  \item \label{item:scst2} Syntactical Completeness: How to verify the language inclusion $\Lang(\TG^S,C^S) \subseteq \Lang(\TGG)^S$ (domain completeness)?
  \item \label{item:scst3} Syntactical Correctness: How to ensure that the resulting graphs of a translation are actually graphs of the target graph language $\Lang(\TG^T,C^T)$, i.e., $\Lang(\TGG)^T \subseteq \Lang(TG^T,C^T)$ (domain correctness)?
  \item \label{item:scst4} Syntactical Completeness \& Correctness: Given verification techniques for the language inclusion problems in \cref{item:scst2,item:scst3}.
  Are the techniques applicable to the derived type graph $\TG$ and the derived set of graph contraints $C$ of \cref{item:scst1}?\envEndMarker
\end{enumerate}
\end{problem}

Beside the completeness problem, \cref{fig:sec-compl-software-trans:softwaretrans} also brings forth the correctness problem of software translations in \cref{prob:scst,item:scst3}.
Therefore, are the resulting graphs 
\begin{enumerate*}
\item models of the target language in text-to-model translations, or
\item actually representations of derivation trees of the target programming language in text-to-text translations such that they can be serialised to target programs?
\end{enumerate*}
While the correctness problem remains subject of future work under the notion of domain correctness, we furthermore concentrate on the completeness problem based on the notion of domain completeness.
While the syntactical completeness of software translations in the graph world itself, i.e., the language inclusion in \cref{fact:comp_soft_trans}, can be verified by means of the existing verification techniques for domain completeness (cf. \cref{prob:scst,item:scst2}), it remains to close the gap between the word grammar and graph world (cf. \cref{prob:scst,item:scst1}) such that the verification techniques for domain completeness are applicable (cf. \cref{prob:scst,item:scst4}).
The gap is closed by presenting a mapping from context-free (EBNF) grammars $G$ to attributed EBNF type graphs $\TG$ together with EBNF graph constraints $C$ in \cref{sec-compl-software-trans-mapping} such that the induced language $\Der(G)$ of derivation trees is isomorphic to the graph language $\Lang(\TG,C)$ of derivation trees (cf. \cref{sec-compl-software-trans-mapping,thm:sec-compl-software-trans:equ_lang_der}).
Furthermore, we use several extensions of the verification techniques for domain completeness in view of the derived EBNF type graphs $\TG$ and EBNF graph constraints $C$ for verifying the completeness of the translation in \cref{sec-compl-software-trans-tgg,ex:sec-compl-software-trans:trans}.
The extension includes
\begin{enumerate*}
\item recursive graph constraints from \cref{sec-dc-general-rec}, and
\item domain restrictions from \cref{sec-dc-general-res}.
\end{enumerate*}
While recursive graph constraints are used for the definition of EBNF graph constraints $C$ in \cref{sec-compl-software-trans-mapping,def:sec-compl-software-trans:ebnf_constraints}, domain restrictions are necessary for the translation in \cref{sec-compl-software-trans-tgg,ex:sec-compl-software-trans:trans} which only covers those parts of a derivation tree which are related to class definitions in a program.
Thus, a restriction of the graph language of derivation trees to a sub-language covering the class definitions only is required for verifying the completeness of the translation, called domain completeness under restrictions (cf. \cref{sec-dc-general-res,def:sec-dc-general-res:dc_prob}).