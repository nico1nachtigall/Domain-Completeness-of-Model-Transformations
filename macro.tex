% \newtheorem{example}{Example}[section]
% \newtheorem{corollary}{Corollary}[section]
% \newtheorem{definition}{Definition}[section]
% \newtheorem{remark}{Remark}[section]
% \newtheorem{proposition}{Proposition}[section]
% \newtheorem{theorem}{Theorem}[section]
% \newtheorem{lemma}{Lemma}[section]
\numberwithin{example}{chapter}
\numberwithin{corollary}{chapter}
\numberwithin{definition}{chapter}
\numberwithin{remark}{chapter}
\numberwithin{proposition}{chapter}
\numberwithin{theorem}{chapter}
\numberwithin{lemma}{chapter}

\newtheorem{fact}{Fact}[chapter]
\newtheorem{problem}{Problem}[chapter]
\newtheorem{claim}{Claim}[chapter]

\newcommand{\mono}{\hookrightarrow}
\newcommand{\monoL}{\hookleftarrow}

% normal text in math mode
\newcommand{\n}[1]{\textnormal{#1}}


\tolerance=10000
\newcommand\xqed[1]{%
  \leavevmode\unskip\penalty9999 \hbox{}\nobreak\hfill
  \quad\hbox{#1}}
\newcommand\envEndMarker{\xqed{$\triangle$}}
\newcommand\envEndMarkerB{ $\triangle$}
\newcommand\envEndMarkerProof{\xqed{$\Box$}}

\newcommand\code[1]{\ensuremath{\mathsf{#1}}}


% Put edit comments in a really ugly standout display
\usepackage{ifthen}
\usepackage{amssymb}
\newboolean{showcomments}
\setboolean{showcomments}{true} % toggle to show or hide comments
\ifthenelse{\boolean{showcomments}}
  {\newcommand{\nb}[2]{
    \fcolorbox{gray}{yellow}{\bfseries%\sffamily
		\scriptsize#1}
    {\sf\small$\blacktriangleright$#2$\blacktriangleleft$}
   }
   \newcommand{\version}{\emph{\scriptsize$-$working$-$}}
  }
  {\newcommand{\nb}[2]{}
   \newcommand{\version}{}
  }

\newcommand\fh[1]{\nb{Frank}{#1}}
\newcommand\nn[1]{\nb{Nico}{#1}}
\newcommand\sg[1]{\nb{Susann}{#1}}

%%%%% TIKZ
\definecolor{myblue}{RGB}{130,40,120}
\usetikzlibrary{positioning,shapes,shadows,arrows}
\tikzstyle{normal}=[rectangle split, rectangle split parts=2,draw=black, text=black, fill=white]
\tikzstyle{box}=[draw=gray]
\tikzstyle{object}=[draw=myblue, rounded corners, fill=black!10, text centered]
\usetikzlibrary{backgrounds}
\usetikzlibrary{shapes.geometric}
\usetikzlibrary{decorations.pathmorphing}
\tikzstyle{background rectangle}=[draw=black!50,fill=yellow!20,rounded corners=1ex]
\tikzstyle{class}=[rectangle, draw=black, rounded corners, fill=white, text centered, text width=2.5cm]



%%%%%%%%%%%%%%%% end TIKZ

%%%%%%%%%%%% cleveref %%%%%%%%%%%
\crefname{theorem}{Thm.}{Thms.} 
\crefname{lemma}{Lem.}{Lemmas} 
\crefname{fact}{Fact}{Facts}
\crefname{example}{Ex.}{Examples}
\crefname{definition}{Def.}{Defs.}
\crefname{corollary}{Cor.}{Corollaries}
\crefname{remark}{Rem.}{Rems.} 
\crefname{figure}{Fig.}{Figs.} 
\crefname{section}{Sec.}{Sects.} 
\crefname{chapter}{Chap.}{Chapters} 
\crefname{proposition}{Prop.}{Props.} 
\crefname{remark}{Rem.}{Remarks.} 
\crefname{part}{Part}{Parts} 
\crefname{enumi}{Item}{Items}%
\crefname{enumii}{Item}{Items}%
\crefname{enumiii}{Item}{Items}%
\crefname{enumiv}{Item}{Items}%
\crefname{enumv}{Item}{Items}%
%%%%%%%%%%%%%%%%%%%%%%%%%%%%%%%%



\newcommand{\longversion}[1]{#1}
\newcommand{\longversionMarker}[1]{%
\longversion{%
%\cbstart%
#1%
%\cbend%
}}
\newcommand{\shortversion}[1]{#1}
\newcommand{\uncuttedversion}[1]{#1}


%\newcommand\envEndMarker{\xqed{\ensuremath{\triangle}}}
\newcommand{\cat}[1]{\ensuremath{\mathbf{#1}}\xspace}
\newcommand{\OP}{\ensuremath{\mathit{OP}}}
\newcommand{\Graphs}{\cat{Graphs}}
\newcommand{\AGraphs}{\cat{AGraphs}}
\newcommand{\ATrGraphs}{\cat{ATrGraphs}}
\newcommand{\Merge}{\ensuremath{\mathrm{Merge}}\xspace}
\newcommand{\Abstr}{\ensuremath{\mathrm{Abstr}}\xspace}
\newcommand{\Inst}{\ensuremath{\mathrm{Inst}}\xspace}
\newcommand{\SRC}{\ensuremath{\mathrm{S}}\xspace}
\newcommand{\C}{\ensuremath{\mathrm{C}}\xspace}
\newcommand{\T}{\ensuremath{\mathrm{T}}\xspace}
\newcommand{\E}{\ensuremath{\mathcal{E}}\xspace}
\newcommand{\EE}{\ensuremath{\mathrm{E}}\xspace}
\newcommand{\M}{\mor{M}}
\newcommand{\morO}{\mor{O}}
\newcommand{\I}{\mor{I}}
\newcommand{\mor}[1]{\ensuremath{{\cal #1}}}
\newcommand{\dom}{\mathrm{dom}}
\newcommand{\ol}[1]{\overline{#1}}
\newcommand{\id}{\ensuremath{\mathit{id}}}

\newcommand{\roundedYellowBoxW}[1]{
\roundedYellowBox{%
		\begin{minipage}{0.98\textwidth}%
		\centering%
		#1%
		\end{minipage}%
}
}

% tikz boxes around figures
\newcommand{\roundedYellowBox}[1]{
	\begin{tikzpicture}[show background rectangle]\node (box)
	{	\hspace*{-6pt}%
			#1%
		\hspace*{-6pt}%
	};
	\end{tikzpicture}%
}



%\renewcommand{\floatpagefraction}{1.0}
%\renewcommand{\topfraction}{1.0}
%\renewcommand{\bottomfraction}{1.0}
%\renewcommand{\textfraction}{0.0}

\renewcommand{\textfraction}{0.0}
\renewcommand{\floatpagefraction}{0.7}
\renewcommand{\topfraction}{1.0}
\renewcommand{\bottomfraction}{1.0}
\setcounter{totalnumber}{5}

% Autoref
\def\sectionautorefname{Sec.}
\def\appendixautorefname{App.}
\def\subsectionautorefname{Sec.}
\def\figureautorefname{Fig.}
\def\theoremautorefname{Thm.}
\def\definitionautorefname{Def.}
\def\remarkautorefname{Rem.}
\def\exampleautorefname{Ex.}
\def\constructionautorefname{Constr.}
\def\factautorefname{Fact}



\newcommand{\timestamp}{
	\com{
		{\small \hfill
		\ddmmyyyydate \renewcommand{\dateseparator}{. }
		\today\\\hfill
		 \currenttime
		 }
	}
}


% ------------------------ Commentary
\newcounter{comCounter}[page]
\setlength{\marginparsep}{2.5mm}
\setlength{\marginparwidth}{2.5cm}
\newcommand{\comE}[1]{
      \normalfont
				\small{\thecomCounter})
                #1				
}

\newcommand{\com}[1]{\stepcounter{comCounter}$^{\thecomCounter)}$
	\marginpar[\flushleft \comE{#1}]{		
        \flushleft
		%\begin{tabular}{|l|}
        \comE{#1}
        %\end{tabular}
		}
	}

% Uncomment the next command to make comments invisible.
%\renewcommand{\com}[1]{}
% --------------------------------END OF Commentary

\renewcommand{\ttdefault}{cmr}

%%%%%%%% Revisions for later Versions (toggle "%")
\usepackage[normalem]{ulem}
\newcommand{\new}[1]
   {\textcolor{blue}{#1}} %in draft version: mark blue
	 % {#1}                 %in final version: black text
\newcommand{\old}[1]
	 {\textcolor{red}{\sout{#1}}} %in draft version: mark red and strike out
	% {}        %in final version: nothing

% -------------------- END OF Commentary





% Derivations
% double-arrow transition
\newcommand{\linefill}{	%
	\cleaders
	\hbox{$\smash{\mkern-2mu\mathord-\mkern-2mu}$}	%
	\hfill
	\vphantom{\lower1pt\hbox{$\rightarrow$}}	%
	}
\newcommand{\Linefill}{%
	\cleaders
	\hbox{$\smash{\mkern-2mu\mathord=\mkern-2mu}$}%
	\hfill \vphantom{\hbox{$\Rightarrow$}}%
} % left segment of extensible arrow, no ending
\newcommand{\TrLinefill}{%
	\cleaders
	\hbox{$\smash{\mkern-2mu\mathord\equiv\mkern-2mu}$}%
	\hfill \vphantom{\hbox{$\Rightarrow$}}%
} % left segment of extensible arrow, no ending
\newcommand{\xmid}[2][]{\stackrel{#2}{\linefill_{\vphantom{#1}}}}
\newcommand{\xMid}[2][]{\stackrel{#2}{\Linefill_{\vphantom{#1}}}} % right segment of extensible arrow, no ending
\newcommand{\xTrMid}[2][]{\stackrel{#2}{\TrLinefill_{\vphantom{#1}}}} % right segment of extensible arrow, no ending
\newcommand{\xleftnoend}[1][]{\mathrel-_{\vphantom{#1}}\mkern-11mu}
\newcommand{\xLeftnoend}[1][]{\mathrel=_{\vphantom{#1}}\mkern-8mu} % left segment of extensible arrow,arrow head
\newcommand{\xTrLeftnoend}[1][]{\mathrel\equiv_{\vphantom{#1}}\mkern-8mu} % left segment of extensible arrow,arrow head
\newcommand{\xrightnoend}[1][]{\mkern-11mu\mathrel-_{\vphantom{#1}}}
\newcommand{\xRightnoend}[1][]{\mkern-8mu\mathrel=_{\vphantom{#1}}} % left segment of
\renewcommand{\xrightarrow}[1][]{\mkern-11mu\rightarrow_{#1}}
\renewcommand{\xleftarrow}[1][]{\leftarrow_{#1}\mkern-11mu}
\renewcommand{\xRightarrow}[1][]{\mkern-8mu\Rightarrow_{#1}} % make arrow symbol
\newcommand{\xRrightarrow}[1][]{\mkern-8mu\Rrightarrow_{#1}} % make arrow symbol
\renewcommand{\xLeftarrow}[1][]{\Leftarrow_{#1}\mkern-8mu} % make arrow symbol
\newcommand{\xmake}[1]{\mathrel{\lower1pt\hbox{$#1$}}} % single-arrow transition
\newcommand{\Trans}[2][]{\xmake{\xLeftnoend[#1]\xMid[#1]{#2}\mkern-2mu\xRightarrow[#1]}}%
\newcommand{\TransMT}{\Rrightarrow}%
%\newcommand{\TransMT}[2][]{\xmake{\xLeftnoend[#1]\xMid[#1]{#2}\mkern-2mu\xRrightarrow[#1]}}%
% \newcommand{\TrTrans}[2][]{\xmake{\xTrLeftnoend[#1]\xTrMid[#1]{#2}\xRrightarrow[#1]}}%
\newcommand\restr[2]{{
  \left.\kern-\nulldelimiterspace 
  #1
  \vphantom{\big|}
  \right|_{#2} 
  }}

%\newcommand{\Trafo}[2]{\Trans{#1}{#2}}

% single-arrow transition
\newcommand{\trans}[2][]{\xmake{\xleftnoend[#1]\xmid[#1]{#2}\xrightarrow[#1]}}%

% single-arrow transition backward
\newcommand{\transB}[2][]{\xmake{\xleftarrow[#1] \xmid[#1]{#2}\xrightnoend[#1]}}%
\newcommand{\TransB}[2][]{\xmake{\xLeftarrow[#1]\xMid[#1]{#2}\xRightnoend[#1]}}%


% Spacing of arrows in xymatrix
\newdir{ >}{{}*!/-10pt/@{>}}
\newdir{->}{{}*!/-18pt/@{>}}
\newdir{ _(}{{}*!/-7pt/@_{(}}
\newdir{ (}{{}*!/-7pt/@^{(}}
\newdir{-(}{{}*!/-15pt/@^{(}}
\newdir{ )}{{}*!/-7pt/@_{(}}
\newdir{>*}{{}*!/-4pt/@{>}*!/-5pt/{^*}}

\newcommand{\ATG}{\ensuremath{\mathit{ATG}}\xspace}
\newcommand{\ATGI}{\ensuremath{\mathit{ATGI}}\xspace}
\newcommand{\PC}{\ensuremath{\mathit{PC}}\xspace}
\newcommand{\AG}{\ensuremath{\mathit{AG}}\xspace}
\newcommand{\GG}{\ensuremath{\mathit{GG}}\xspace}
\newcommand{\AS}{\ensuremath{\mathit{AS}}\xspace}
\newcommand{\Att}{\ensuremath{\mathit{Att}}\xspace}
\newcommand{\SG}{\ensuremath{\mathit{SG}}\xspace}

\newcommand{\Lang}{\ensuremath{\mathcal{L}}\xspace}
\newcommand{\PILTOSPELL}{\ensuremath{\mathit{PIL2SPELL}}\xspace}
\newcommand{\TRAFOS}{\ensuremath{\mathit{TRAFOS}}\xspace}
\newcommand{\ac}{\ensuremath{\mathit{ac}}\xspace}
\newcommand{\AC}{\ensuremath{\mathit{AC}}\xspace}
\newcommand{\tExt}{\ensuremath{\mathit{tExt}}\xspace}
\newcommand{\acext}{\ensuremath{\mathit{Ext}}\xspace}
\newcommand{\inc}{\ensuremath{\mathit{inc}}\xspace}
\newcommand{\CLG}{\ensuremath{\mathit{CLG}}\xspace}
\newcommand{\CG}{\ensuremath{\mathit{CG}}\xspace}
\newcommand{\LTS}{\ensuremath{\mathit{LTS}}\xspace}
\newcommand{\IO}{\ensuremath{\mathit{IO}}\xspace}
\newcommand{\DC}{\ensuremath{\mathit{DC}}\xspace}
\newcommand{\DG}{\ensuremath{\mathit{DG}}\xspace}
\newcommand{\DCG}{\ensuremath{\mathit{DCG}}\xspace}
\newcommand{\TGG}{\ensuremath{\mathit{TGG}}\xspace}
\newcommand{\TS}{\ensuremath{\mathit{TS}}\xspace}
\newcommand{\MT}{\ensuremath{\mathit{MT}}\xspace}
\newcommand{\fPpg}{\ensuremath{\mathit{fPpg}}\xspace}
\newcommand{\bPpg}{\ensuremath{\mathit{bPpg}}\xspace}
\newcommand{\Synch}{\ensuremath{\mathit{Synch}}\xspace}
\newcommand{\MSynch}{\ensuremath{\mathit{MSynch}}\xspace}
\newcommand{\proj}{\ensuremath{\mathit{proj}}\xspace}
\newcommand{\False}{\ensuremath{\mathbf{F}}\xspace}
\newcommand{\True}{\ensuremath{\mathbf{T}}\xspace}
%\newcommand{\Att}{\ensuremath{\mathit{Att}}\xspace}
\newcommand{\TR}{\ensuremath{\mathit{TR}}\xspace}
\newcommand{\VL}{\ensuremath{\mathit{VL}}\xspace}
\newcommand{\TG}{\ensuremath{\mathit{TG}}\xspace}
\newcommand{\tr}{\ensuremath{\mathit{tr}}\xspace}
\newcommand{\FT}{\ensuremath{\mathit{FT}}\xspace}
\newcommand{\BT}{\ensuremath{\mathit{BT}}\xspace}
\newcommand{\true}{\textbf{true}\xspace}
\newcommand{\false}{\textbf{false}\xspace}
\newcommand{\TGT}{\textsc{TGT}\xspace}
\newcommand{\MM}{\ensuremath{\mathit{MM}}\xspace}
\newcommand{\UD}{\ensuremath{\mathit{UD}}\xspace}
\newcommand{\Ppg}{\ensuremath{\mathit{Ppg}}\xspace}
\newcommand{\PPG}{\ensuremath{\mathit{PPG}}\xspace}
\newcommand{\CC}{\ensuremath{\mathit{CC}}\xspace}

\newcommand{\aln}{\ensuremath{\mathit{Aln}}\xspace}
\newcommand{\del}{\ensuremath{\mathit{Del}}\xspace}
\newcommand{\fAln}{\ensuremath{\mathit{fAln}}\xspace}
\newcommand{\fAdd}{\ensuremath{\mathit{fAdd}}\xspace}
\newcommand{\MF}{\ensuremath{\mathit{MF}}\xspace}
\newcommand{\add}{\ensuremath{\mathit{Add}}\xspace}
\newcommand{\ext}{\ensuremath{\mathit{Ext}}\xspace}
\newcommand{\marking}{\ensuremath{\mathit{mark}}\xspace}
\newcommand{\markinguc}{\ensuremath{\mathit{Mark}}\xspace}
\newcommand{\ob}{\ensuremath{\mathit{Ob}}\xspace}
\newcommand{\morB}{\ensuremath{\mathit{Mor}}\xspace}
\newcommand{\fin}{\ensuremath{\mathit{fin}}\xspace}
\newcommand{\TF}{\ensuremath{\mathit{\mathcal{TF}}}\xspace}
\newcommand{\LHS}{\ensuremath{\mathit{LHS}}\xspace}
\newcommand{\RHS}{\ensuremath{\mathit{RHS}}\xspace}
\newcommand{\NAC}{\ensuremath{\mathit{NAC}}\xspace}
\newcommand{\Der}{\ensuremath{\mathit{Der}}\xspace}
\newcommand{\Rel}{\ensuremath{\mathit{Rel}}\xspace}
\newcommand{\GS}{\ensuremath{\mathit{GS}}\xspace}
\newcommand{\CD}{\ensuremath{\mathit{CD}}\xspace}
\newcommand{\RDBM}{\ensuremath{\mathit{RDBM}}\xspace}
\newcommand{\Corr}{\ensuremath{\mathit{Corr}}\xspace}
\newcommand{\Fin}{\ensuremath{\mathit{Fin}}\xspace}
\newcommand{\Paths}{\ensuremath{\mathit{Paths}}\xspace}
\newcommand{\paths}{\ensuremath{\mathit{path}}\xspace}
\newcommand{\LA}{\ensuremath{\mathit{LA}}\xspace}
\newcommand{\type}{\ensuremath{\mathit{type}}\xspace}
\newcommand{\Pset}{\mor{P}}
\newcommand{\DSIG}{\ensuremath{\mathit{DSIG}}\xspace}
\newcommand{\Restr}{\ensuremath{\mathit{Restr}}\xspace}
\newcommand{\der}{\ensuremath{\mathit{der}}\xspace}
\newcommand{\NC}{\ensuremath{\mathit{NC}}\xspace}
\newcommand{\iter}{\ensuremath{\mathit{it}}\xspace}
\newcommand{\cond}{\ensuremath{\mathit{cond}}\xspace}
\newcommand{\GTS}{\ensuremath{\mathit{GTS}}\xspace}
\newcommand{\EBNF}{\ensuremath{\mathit{EBNF}}\xspace}

\newcommand{\comBlockArgs}[2]{
\noindent\fbox{~~TODO~~}\fbox{#1}
\\
\noindent\fbox{\parbox{\columnwidth}{#2}} }

\newcommand{\comBlock}[1]{
\noindent\fbox{~~TODO~~}
\\
\noindent\fbox{\parbox{\columnwidth}{#1}} }

\newcommand{\quotient}[2]{\raisebox{0em}{$#1$}\raisebox{-.2em}{$\mid$}\raisebox{-.4em}{$#2$}}




\definecolor{darkblue}{rgb}{0,0,0.6}
\definecolor{darkred}{rgb}{0.6,0,0}
\definecolor{darkgreen}{rgb}{0,0.5,0.2}
\definecolor{grey}{rgb}{0.5,0.5,0.5}
\definecolor{darkgrey}{rgb}{0.3,0.3,0.3}
\definecolor{lightgrey}{rgb}{0.95,0.95,0.95}
\definecolor{violett}{rgb}{0.5,0,0.33}

\lstdefinelanguage{pseudocode}{
  basicstyle=\footnotesize,
  linewidth=14cm
}

\lstset{ %
  linewidth=7cm,
  backgroundcolor=\color{white},   % choose the background color; you must add \usepackage{color} or \usepackage{xcolor}
  basicstyle=\scriptsize,        % the size of the fonts that are used for the code
  breakatwhitespace=false,         % sets if automatic breaks should only happen at whitespace
  breaklines=true,                 % sets automatic line breaking
  postbreak=\raisebox{0ex}[0ex][0ex]{\ensuremath{\color{black}\hookrightarrow\space\space\space}},
  captionpos=b,                    % sets the caption-position to bottom
  deletekeywords={...},            % if you want to delete keywords from the given language
  escapeinside={\%*}{*)},          % if you want to add LaTeX within your code
  extendedchars=true,              % lets you use non-ASCII characters; for 8-bits encodings only, does not work with UTF-8
  frame=single,	                   % adds a frame around the code
  keepspaces=true,                 % keeps spaces in text, useful for keeping indentation of code (possibly needs columns=flexible)
  %keywordstyle=\color{blue},       % keyword style
  %language=Octave,                 % the language of the code
  otherkeywords={class,:,new,print,read,end,if,then,goto,null,terminal},           % if you want to add more keywords to the set
  numbers=left,                    % where to put the line-numbers; possible values are (none, left, right)
  numbersep=5pt,                   % how far the line-numbers are from the code
  %numberstyle=\tiny%\color{mygray}, % the style that is used for the line-numbers
  showspaces=false,                % show spaces everywhere adding particular underscores; it overrides 'showstringspaces'
  showstringspaces=false,          % underline spaces within strings only
  showtabs=false,                  % show tabs within strings adding particular underscores
  stepnumber=2,                    % the step between two line-numbers. If it's 1, each line will be numbered
  tabsize=2,	                   % sets default tabsize to 2 spaces
  title=\lstname                   % show the filename of files included with \lstinputlisting; also try caption instead of title
}
