According to \cref{sec-gen-intro-mt}, triple graph grammars (TGGs) allow the definition of visual, declarative transformation specifications for model transformations and synchronisations.
Therefore, we review basic notions of TGGs from \cite{FAGT2}.

A triple graph\index{triple graph} $G=(G^\SRC \transB{s_G} G^\C \trans{t_G} G^\T)$ is an integrated model consisting of a source graph $G^\SRC$, a target graph $G^\T$ and explicit correspondences between them (cf. Def. 3.3 in \cite{FAGT2}).
The correspondences are given by correspondence graph $G^\C$ together with morphisms $s_G\colon G^\C \to G^\SRC$ and $t_G\colon G^\C \to G^\T$ specifying a correspondence relation between elements of $G^\SRC$ and elements of $G^\T$.
Triple graphs $G$ and $H$ are related by triple graph morphisms\index{triple graph!morphism} $m=(m^\SRC,m^\C,m^\T) : G \rightarrow H$~\cite{DBLP:conf/wg/Schurr94,DBLP:conf/fase/EhrigEEHT07} consisting of three morphisms $m^\SRC\colon G^\SRC \to H^\SRC,m^\C\colon G^\C \to H^\C$ and $m^\T\colon G^\T \to H^\T$ that preserve the associated correspondences, i.e., $m^\SRC \circ s_G=s_H \circ m^\C$ and $m^\T \circ t_G=t_H \circ m^\C$.  
Therefore, analogously to attributed graphs and morphisms in \cref{sec-gt-graphs,def:sec-gt-graphs:agraphs}, an attributed triple graph $G=(G^\SRC \transB{s_G} G^\C \trans{t_G} G^\T)$ is defined by attributed graphs $G^\SRC,G^C,G^\T$ and attributed graph morphisms $s_G,t_G$.
An attributed triple graph morphism $f=(f^\SRC,f^\C,f^\T)\colon G \rightarrow H$ between two attributed triple graphs $G$ and $H$ is defined by three attributed graph morphisms $f^\SRC,f^\C,f^\T$.
Furthermore, analogously to typed attributed graphs and morphisms (with node type inheritance) in \cref{sec-gt-graphs,def:sec-gt-graphs:typed_attr_graphs,rem:sec-gt-graphs:inheritance}, a typed attributed triple graph\index{triple graph!typed \& attributed} over attributed triple graph $\TG$ as attributed triple type graph is defined by an attributed triple graph $G$ together with an attributed triple graph morphism $\type_G\colon G \to \TG$.
A typed attributed triple graph morphism\index{triple graph!morphism!typed \& attributed} $f\colon G \rightarrow H$ between typed attributed triple graphs $G$ and $H$ is an attributed triple graph morphism $f$ such that $(\type_H^X \circ f^X=\type_G^X)_{X \in \{\SRC,\C,\T\}}$.
The category $(\ATrGraphs_\ATGI,\M)$\index{category!$\ATrGraphs_\ATGI$} of all typed attributed triple graphs over triple type graph $\ATGI$ and all typed attributed triple graph morphisms with node type inheritance is $\M$-adhesive where according to \cref{sec-gt-M-adh,rem:sec-gt-M-adh:agraphs_atgi}, triple graph $\M$-($\morO$-)morphisms\index{category!$\ATrGraphs_\ATGI$!$\M$-morphisms}\index{category!$\ATrGraphs_\ATGI$!$\morO$-morphisms} are componentwise $\M$-($\morO$-) morphisms in $(\AGraphs_\ATGI,\M)$ (cf. Def. 3.4 \& Thm. 7.2 in \cite{FAGT2}).
Analogously to the definition of rules in \cref{sec-gt-trafo,def:sec-gt-trafo:rule} and according to \cite{GEH11} and Def. 3.8 in \cite{FAGT2}, a triple rule\index{triple rule} $\tr=(\ol{\tr}\colon L \xhookrightarrow{} R,\ac_L)$ is given by a triple graph $\M$-morphism $\ol{tr}$ and an application condition $\ac_L$ over $L$.
Thus, triple rules are non-deleting and specify how a given consistently integrated model (triple graph) can be extended simultaneously on all three components source, correspondence and target yielding again a consistently integrated model. 
Analogously to (direct) transformations in \cref{sec-gt-trafo,def:sec-gt-trafo:trafo}, for a given triple graph $G$, triple rule $\tr=(\ol{\tr}\colon L \xhookrightarrow{} R,\ac_L)$ and triple graph match-morphism $m\colon L \to G$ with $m \models \ac_L$, (direct) triple graph transformations\index{triple graph!transformation}\index{triple graph!transformation!direct} $G \Trans{(\tr,m)} H$ via $\tr$ and $m$ are defined (cf. Def. 3.8 in \cite{FAGT2}).
Analogously to grammars in \cref{sec-gt-trafo,def:sec-gt-trafo:grammar}, a triple graph grammar $\TGG=(S,\TR)$\index{triple graph!grammar (TGG)} consists of a triple start graph $S$ and a set $\TR$ of triple rules, and generates the triple graph language of consistently integrated models $\Lang(\TGG)=\{G \mid \exists \text{ triple graph transformation } S \Trans{*} G \text{ via }\TR\}$\index{triple graph!grammar (TGG)!language $\Lang(\TGG)$} with consistent source and target languages $\Lang(\TGG)^\SRC=\{G^\SRC \mid (G^\SRC \leftarrow G^\C \to G^\T) \in \Lang(\TGG) \}$ and $\Lang(\TGG)^\T=\{G^\T \mid (G^\SRC \leftarrow G^\C \to G^\T) \in \Lang(\TGG) \}$\index{triple graph!grammar (TGG)!language $\Lang(\TGG)^\SRC$}\index{triple graph!grammar (TGG)!language $\Lang(\TGG)^\T$} (cf. Def. 3.12 in \cite{FAGT2}).

\paragraph*{Visual Notation}
As depicted in \cref{sec-gt-trafo,fig:sec-gt-trafo:tgg}, the three components of triple graphs are visualised in three separate boxes.
According to the conventions for the visual notation of graphs in \cref{par:sec-gt-graphs:vis}, the mapping of graph elements along correspondence morphisms $s_G$ and $t_G$ of triple graphs $G$ correspond to their naming in visual notation.
Additionally, the conventions for the visual notation of rules in \cref{par:sec-gt-trafo:vis_not} are also used for triple rules.
For example, each triple rule in \cref{sec-gt-trafo,fig:sec-gt-trafo:tgg} visualises the different domains of UML class diagrams (left boxes), correspondences (boxes inbetween) and relational database models (right boxes) in separate boxes.
Furthermore, each rule adds those graph elements that are marked with \code{++} while all unmarked elements are preserved when being applied.
Moreover, for example in rule \code{3} node \code{1:CT} is mapped to nodes \code{1:Class} and \code{1:Table} along the correspondence morphisms, respectively.

\begin{example}[Triple Graph, Triple Type Graph \& Triple Graph Grammar (CD2RDBM)]
\label{ex:sec-tgg:tg}
Attributed triple graph $G=(\CD \transB{s} C \trans{t} \RDBM)$ in \cref{sec-gt-graphs,fig:sec-gt-graphs:atg} is typed over attributed triple type graph $\TG=(\TG_\CD \gets \TG_C \to \TG_\RDBM)$ via attributed triple graph type morphism $(\type_\CD,\type_C,\type_\RDBM)\colon G \to \TG$.
According to the type graph, each \code{Attr}ibute in CDs correspond to a \code{Column} in RDBMs via a node of type \code{AC}, each \code{Class} in CDs correspond to a \code{Table} in RDBMs via a node of type \code{CT}, and each \code{DataType} in CDs correspond to a \code{ColumnType} in RDBMs via a node of type \code{TT}.
Therefore, node \code{1:DataType} in source graph $\CD$ corresponds to node \code{1:ColumnType} in target graph $\RDBM$ via node \code{1:TT} in correspondence graph $C$, node \code{2:Attr} in $\CD$ corresponds to node \code{1:Column} in $\RDBM$ via node \code{2:AC} in $C$, and node \code{4:Class} in $\CD$ corresponds to node \code{4:Table} in $\RDBM$ via node \code{4:CT} in $C$.
\cref{sec-gt-trafo,fig:sec-gt-trafo:tgg} depicts the triple rules for creating consistently integrated models of UML class diagrams (CDs) together with corresponding relational database models (RDBMs).
The rules are all typed over $\TG$.
Given a triple graph with class diagram $\CD'$ in the source and database model $\RDBM'$ in the target such that $\CD'$ contains a \code{DataType} with corresponding \code{ColumnType} in $\RDBM'$, then the application of triple rule \code{1} extends the triple graph simultaneously on all three components in the sense that it simultaneously adds a \code{Class} to $\CD'$ together with a corresponding \code{Table} to $\RDBM'$ both of \code{name} \code{n} but only if $\CD'$ does not already contain a class of the same name (cf. NAC).
Furthermore, the table is equipped with a dedicated \code{Column} as primary key (\code{pkey}) of \code{type} \code{ColumnType}.
Rule \code{2} simultaneously adds a \code{DataType} to CDs together with a corresponding \code{ColumnType} to RDBMs, both of \code{name} \code{n}.
Rule \code{3} simultaneously adds an \code{Attr}ibute to CDs together with a corresponding \code{Column} to RDBMs, both of \code{name} \code{n}, and assigns both to an existing class in CDs and the corresponding table in RDBMs.
Rule \code{4} adds a \code{Const}ant modifier to an existing attribute in CDs but only if the attribute does not already have a modifier (cf. NAC).
Rule \code{5} simultaneously assigns an existing data type to an existing attribute as type in CDs and the corresponding column type to the corresponding column as type in RDBMs but only if the column does not already have a type (cf. NAC).
Rule \code{6} simultaneously assigns an existing class to an existing attribute as type in CDs and the corresponding table to the corresponding column as foreign key (\code{fkey}) in RDBMs but only if the column does not already have a type (cf. NAC), i.e., the type of the primary key column (\code{pkey}) of the table is additionally assigned to the column as type.
The TGG $CD2RDBM=(\varnothing,\{\code{1},\code{2},\code{3},\code{4},\code{5},\code{6}\})$ for transforming CDs into RDBMs is given by the empty triple start graph $\varnothing$ together with triple rules \code{1} to \code{6}, i.e., the TGG is typed over $\TG$.
Triple graph $G=(\CD \transB{s} C \trans{t} \RDBM)$ in \cref{sec-gt-graphs,fig:sec-gt-graphs:atg} can be obtained via direct triple graph transformations $\varnothing \Trans{(\code{2},\_)} G_1 \Trans{(\code{1},\_)} G_2 \Trans{(\code{3},\_)} G_3 \Trans{(\code{4},\_)} G_4 \Trans{(\code{5},\_)} G$ via triple rules $\code{1}$ to $\code{5}$, i.e., $G \in \Lang(CD2RDBM),\CD \in \Lang(CD2RDBM)^\SRC$ and $\RDBM \in \Lang(CD2RDBM)^\T$.
\envEndMarker
\end{example}

\begin{remark}[Meta-Modelling \& Model Transformation]
As discussed in \cref{sec-gen-intro-mt}, a model transformation between DSLs $\Lang(D_1)$ and $\Lang(D_2)$ transforms models from language $\Lang(D_1)$ in source domain $D_1$ to language $\Lang(D_2)$ in target domain $D_2$ where each DSL is defined by a meta-model in the corresponding domain.
In the given context of graph transformations, a meta-model is defined by a type graph together with a set of graph constraints.
Therefore, a DSL $\Lang(D)$ in domain $D$ is given by all graphs that are typed over the domain type graph and that satisfy the domain constraints (cf. \cref{sec-dc-general,def:sec-dc-general:lang}).
The attributed triple type graph $\TG=(\TG_\CD \gets \TG_C \to \TG_\RDBM)$ in \cref{sec-gt-graphs,fig:sec-gt-graphs:atg} contains both the type graph $\TG_\CD$ for the domain of class diagrams (CDs) and type graph $\TG_\RDBM$ for the domain of relational database models (RDBMs) together with type graph $\TG_C$ for correspondences between both.
Additionally, \cref{sec-gt-gc,fig:sec-gc-gc:CD_constraints} represents the graph constraints for the domain of CDs (constraints for RDBMs can be defined analogously).
Thus, for the model transformation CD2RDBM from CDs to RDBMs and in view of \cref{sec-gen-intro-mt,fig:sec-gen-intro-msynch:mt_msynch} (a) and (c), a graph $M$ ($M'$) conforms to the meta-model in domain CD (RDBM), if $M$ ($M'$) is typed over $\TG_\CD$ ($\TG_\RDBM$) and satisfies the CD (RDBM) constraints.
Graph $\CD$ in \cref{fig:sec-gt-graphs:atg} conforms to the CD meta-model and graph $\RDBM$ conforms to the RDBM meta-model.
The transformation language of CD2RDBM is given by the formalism of TGGs and contains all TGGs that conform to the meta-models in domains CD and RDBM, i.e., all TGGs that are typed over triple type graph $\TG$ and that satisfy the domain constraints.
According to \cref{ex:sec-tgg:tg}, the TGG CD2RDBM in \cref{sec-gt-trafo,fig:sec-gt-trafo:tgg} is in (conforms to) the transformation language of CD2RDBM and therefore, is a valid transformation specification.
Transformation CD2RDBM may take graph $\CD$ in \cref{fig:sec-gt-graphs:atg} as input and outputs triple graph $(\CD \transB{s} C \trans{t} \RDBM)$ with correspondences $C$ if being executed based on the TGG in \cref{fig:sec-gt-trafo:tgg}.
In \cref{sec-mt-tgg,def:sec-mt-tgg:mt_ft}, we review the execution of transformations based on model transformation sequences and a given TGG in more detail.
\envEndMarker
\end{remark}
